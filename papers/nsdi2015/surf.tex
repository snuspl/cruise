\documentclass[twocolumn,10pt]{article}
\usepackage[usenix,nocopyright,10pt]{sigmin}

\usepackage{endnotes,verbatim}
%\usepackage{setspace}
%\doublespacing
 
% Page layout and spacing
%\usepackage[margin=1in]{geometry}
%\usepackage[medium,compact]{titlesec}

% Formatting
%\usepackage{alltt}
\usepackage{url}
% Type1 fonts please!
\usepackage[T1]{fontenc}
%\usepackage{times,courier,mathptmx}
\usepackage{times}
\usepackage{textcomp}

\usepackage[tight]{subfigure}
\usepackage{graphicx}
\usepackage{amsmath}
\usepackage{epsfig}

\usepackage{cite}
\usepackage{color}
\usepackage{xspace}

\newtheorem{theorem}{Theorem}[section]
\newtheorem{proof}{Proof}[section]
\newtheorem{lemma}[theorem]{Lemma}

\newtheorem{proposition}[theorem]{Proposition}
\newtheorem{corollary}[theorem]{Corollary}
\newtheorem{definition}{Definition}

%\newcommand{\note}[1]{}
\newcommand{\note}[1]{[\textcolor{red}{\textit{#1}}]}
\newcommand{\msg}[1]{{\textsc{\small #1}}}
\newcommand{\method}[1]{{\texttt{\small #1}}}
\newcommand{\stitle}[1]{\vspace{.5pt}{\noindent\bf #1:}}

\newcommand{\lbft}{LBFT\xspace}
\newcommand{\lbftparen}{(LBFT)\xspace}

\newcommand{\eat}[1]{}

\title{Surf}

\author{XYZ}

%\date{}

% make the bibliography compact
\let\oldthebibliography=\thebibliography
\let\endoldthebibliography=\endthebibliography
\renewenvironment{thebibliography}[1]{%
    \begin{oldthebibliography}{#1}%
    \setlength{\parskip}{0ex}%
    \setlength{\itemsep}{0ex}%
}%
{%
    \end{oldthebibliography}%
}

\begin{document}

\maketitle

\begin{abstract}


\end{abstract}

\section{Introduction}

\begin{itemize}
\item Why in-memory caching tier for Big Data Analytics?

\begin{itemize}
\item Performance gains of in-memory processing e.g., Spark.
\item Big Data workloads are PBs of data, still requiring disks to play a big role in computations e.g., Hadoop MR.
\item Many application-level requirements are a combination of the two (need examples and/or references).
But each application has its own memory/disk space and performance tradeoffs.
Hard to provision memory resources before-hand, and across multiple frameworks.
\item Need a flexible in-memory layer that allows applications to make use of scarce memory resources.
\end{itemize}

\item Our new approach: a generic, transparent caching service backed by a backend distributed file system \\

cache: soft-state\\

support for multi-frameworks\\

how file system clients use surf? Use surf:// instead of hdfs:// in their file system call, and the rest of things are handled by surf.

\item Unique features of Surf
\begin{itemize}
\item Application-driven data replication : per-file replication policy
\item Elastic cluster-level memory pool provisioning: increase or decrease the number of caching servers depending on caching load
\end{itemize}

\item Implementation: built upon REEF~\cite{reef}, lines of code; 
In-production use, contributed to Apache

\item Evaluation: how Surf beats HDFS, HDFS caching

\item Paper roadmap

\end{itemize}


\section{Overview}

\noindent We need a nice transition between Intro and Design

\begin{itemize}
\item Maybe an overview of Surf?
\item With a good motivating application (SKT?) we could give a brief example of it running with Surf. Compare it to running without Surf.
\end{itemize}

\section{Surf Design}

\begin{itemize}
\item Basic Design - non-elastic, non-replication version
\begin{itemize}
\item Pinning
\item Cache eviction
\end{itemize}
\item Adding flexible replication - cache all, cache one, cache a few 
\item Adding elasticity
\item Discussion?
\end{itemize}


\section{Implementation}

Surf is built atop REEF\cite{reef}.

\begin{itemize}
\item Lines of code
\item Client library exposing the file system interface, Thrift between client and caching task
\item Server
\end{itemize}

\section{Evaluation}

\noindent Experiment setup - how many nodes, the spec. of each node

\noindent What do we compare with Surf: HDFS, HDFS caching

\noindent Summary of what we want to show from the experiments

\begin{itemize}
\item Microbenchmarks - explain
\begin{itemize}
\item Result 1 (Each graph should make a point. )
\item Result 2 
\item ...
\end{itemize}

\item Macrobenchmarks - Hadoop MR jobs
\begin{itemize}
\item Result 1
\item Result 2
\item ...
\end{itemize}

\item Macrobenchmarks - SKT workloads (Shark/Spark jobs)  
\begin{itemize}
\item Result 1
\item Result 2
\item ...
\end{itemize}

\end{itemize}

\section{Related Work}

\begin{itemize}
\item In-memory work in Big Data Analytics
\begin{itemize}
\item HDFS caching: tied to OS page cache, not elastic, not flexible
\item Spark RDD~\cite{sparknsdi}: tied to a particular framework, surf: independent of frameworks
\item PacMan~\cite{pacman}
\item Tachyon - in-memory file system~\cite{tachyon}: only a single copy in memory, file system semantics, recovery etc., surf: soft-state
\end{itemize}
\item Caching in other domains: Memcache, Web caching, CDN, etc.
\end{itemize}

\section{Conclusion and Future Work}

\begin{itemize}
\item Contribution
\item Summary of the numbers
\item Say it's in production use in SKT
\item Say it's contributed to the Apache incubation project
\item Future work
\begin{itemize}
\item write path? - anything we do to improve write performance?
\item How to handle intermediate data (written) that's not backed by the distributed file system
\item running code in the same JVM that hosts the cached data,
\end{itemize}

\end{itemize}

\bibliographystyle{abbrv}
\bibliography{surf}

\end{document}




